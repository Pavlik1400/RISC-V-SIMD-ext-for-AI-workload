%----------------------------------------------------------------------------------------
%	Introduction
%----------------------------------------------------------------------------------------

\chapter{Introduction}
Machine learning (ML) has become increasingly popular and useful in recent years due to its ability to handle complex tasks and make accurate predictions. It is used for computer vision, natural language processing, speech recognition, robotics, and so on~\cite{ml_applications}. One of the critical areas where ML models have shown great promise is signal modulation classification, an important problem in wireless communication systems~\cite{darpa_ml_application}. Accurately classifying the modulated signals in a communication channel is crucial for some wireless communication, military, and IoT systems. 

Unfortunately, most ML models today are computationally intensive~\cite{models_require_alotf_of_computations}, and the traditional processing units may not be efficient enough to perform these computations effectively. Cloud computing is a widespread approach to using complex models that require much computation power on simple and cheap systems. But it has many problems related to latency, reliability, communication bottleneck, privacy, etc. That is why there is a growing need for edge computing. Edge computing allows for processing and storing data closer to the source, reducing the need to transmit data back and forth from a centralized server~\cite{edge_computing}. Custom Function Units (CFUs) have emerged as a promising solution~\cite{risc_v_cfu}. CFUs are specialized processing units that can be integrated into a general-purpose processor to accelerate the execution of specific tasks. They are designed to perform specific operations and can be programmed to suit the application's specific needs. 

The rise of open-source hardware and processors like RISC-V~\cite{risc_v_manual} has made developing custom function units for edge computing much easier and gives more flexibility to the design since the CPU can also be configured for a concrete application. Field Programmable Gate Arrays (FPGA) today are powerful enough to fit such processors, significantly speeding up, simplifying, and making cheaper experimenting with such hardware designs. Frameworks provided by FPGA manufacturers and the open-source community help to optimize custom hardware for latency and energy efficiency. 

Modern ML frameworks, like Tensorflow Lite Micro (TFLM) and PyTorch Mobile~\cite{tflm_and_pytorch_mobile} are highly optimized for limited hardware. They can deploy models on embedded systems fairly simply, and the open-source nature of these frameworks enables developers to seek for computation bottleneck of the deployed models. CFU-Playground~\cite{cfu_playground} allows to unite listed above technologies -- it allows CFU development for VexRiscV~\cite{vexriscv} soft processor that implements RISC-V ISA and CFU extension and runs TFLM model(s) on it. The resulting model can be executed on both the Renode simulator~\cite{renode} and synthesized on the FPGA board. 

Therefore, in this bachelor thesis, we propose using a CFU in a RISC-V processor to accelerate the execution of the Convolutional Neural Network (CNN) used for signal modulation classification. Different architectures and datasets are considered to train such a model. We investigate deployed model to find the execution bottlenecks that CFU can accelerate. We investigate the design and implementation of the CFU and its integration with the RISC-V processor. We also evaluate the performance of the proposed system in both simulation and FPGA boards. The results show that the proposed system can significantly reduce execution time and power consumption while maintaining high accuracy in signal modulation classification. This research could contribute to developing more efficient and accurate wireless communication systems and enable deploying edge computing systems in resource-constrained environments.

\section{Structure}
\begin{itemize}
    \item Chapter~\ref{Chapter 2} describes the signal modulation classification problem and reviews current classical and ML approaches to the problem. Different acceleration approaches are described. RISC-V ISA and FPGA are overviewed.
    \item Chapter~\ref{Chapter 3} introduces technologies for developing accelerators and the development pipeline in CFU-Playground.
    \item Chapter~\ref{Chapter 4} describes models we developed for the modulation classification problem.
    \item Chapter~\ref{Chapter 5} explains developed accelerators, presents results of model acceleration, comparison of embedded and simulated systems, measurements of inference execution time, and utilization of FPGA resources.
    \item Chapter~\ref{Chapter 6} concludes the thesis with a summary of the findings and discusses further work.
    \item The work includes the following appendixes:~\cref{Appendix_A,Appendix_B,Appendix_C,Appendix_D} 
\end{itemize}